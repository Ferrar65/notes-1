\documentclass{article}

\usepackage{amsmath, mathtools, amsthm}
\usepackage{graphicx}
\graphicspath{ {./images/} }

\title{Positivismo}
\author{github.com/asdrubalini}
\date{\today}

\begin{document}
    \maketitle

    \section{Positivismo}
    Periodo nel quale si afferma una sorta di nuova religione.\\
    L'uomo diventa artefice indiscusso del proprio progresso.
    Tutto ciò che riguarda la metafisica non deve essere oggetto dello studio dell'indagine filosofica. La filosofia mette in discussione la ragione.

    Secondo Comte:
    
    "Basta occuparsi di metafisica, basta pensare ai massimi sistemi. Indagine filosofica deve essere un'indagine scientifica. No questioni indecifrabili. Più argomenti accessibili che riguardano il reale.".\\
    Dovrebbe cercare il miglioramento delle condizioni dell'uomo.\\
    Secondo lui la filosofia sono inutili dubbi.\\
    Nel pensiero dell'uomo esiste una sorta di evoluzionismo Darwiniano.

    Speculazione, ma non quella economica.
    La filosofia serve per dare certezze, e le certezze le da la scienza e la tecnica.

    Comte - Darwin - Nietzsche

    L'uomo è profondamente determinato a seconda dell'ambiente in cui vive.\\
    L'uomo ha il gusto per il dolce perchè lo portiamo nel DNA che affonda le sue origini dalla preistoria.\\
    Due entità nella storia: individuo e società.

    Nietzsche: "Non esiste una razionalità nella storia."\\
    Superuomo per Nietzsche è colui che dopo aver lottato ed essersi imposto, regnerà sulla terra. Dominerà la terra ed i soggetti più deboli dopo la morte di Dio. La fine dei vecchi valori. Imparerà a liberarsi dalla morale. Non esiste più il bene e il male. Autodeterminazione dell'uomo ed una conseguente affermazione del diritto di questo superuomo di fare quello che vuole.

    \section{Corrente artistica della scapigliatura}

    17 marzo 1861: proclamazione del regno d'Italia -> V. Emanule II Re d'Italia
    Destra storica al potere. Il suffraggio era mini censitario. Al potere c'era il partito liberale, ovvero la destra storica, fino al 1876. Si radunavano a Torino e decisero la fine dell'Italia.

    I problemi dell'Italia:
    Unificazione italiana:

    \begin{itemize}
        \item Trentino alto adige e Friuli Venezia-Giulia, prima guerra mondiale
        \item Veneto, conquistato con la terza guerra di indipendenza nel 1866
        \item Roma e Lazio, nel 20 settembre 1870
    \end{itemize}

    Debito economico: protratto dal Piemonte per la sua campagna militare pagata da tutti con numerose tasse che colpivano il popolino. Tassa del macinato: sulla farina al mulino. Vuoi mangiare il pane? Paghi la tassa.

    Questione meridionale.

    Scrive Massari: il contadino si vede e si sente condannato a perpetua miseria, e subisce antiche e secolari ingiustizie. Ma non si fa nulla per risolvere il problema.

    2 Settembre 1970 a Sedan: battaglia in cui Napoleone III viene sconfitto e Parigi diventa comunista.

    Lo Stato viene visto come nemico, per colpa della leva militare, delle troppe tasse, dello statuto albertino esteso a tutti... e così il sud iniziò a rivolgersi alla mafia.

    "Noi siamo figli di padri ammalati, aquile al tempo di buttar le piume". La malattia per gli scapigliati è la tristezza romantica.

    Non sopportavano in alcun modo Manzoni. Si impongono contro ogni forma di cristianesimo. Ripudiavano la società in cui vivevano, fondata sul materialismo consumistico. Erano figli di borghesi ma ripudiavano la società borghese. Erano anticonformisti. Vita anarchica e sregolata come era la loro arte. Morirono tutti prematuramente.

    La poesia doveva essere indipendente e senza alcuna finalità educativa.

    I massimi esponenti:
    Giovanni Camerana
    Emilio Praga
    Virgilio Ugo Tarchetti

    \section{Il verismo}
    Movimento letterale che si afferma in italia dal 1860 in poi. Finisce nel 1889.
    Giovanni Verga

    Punti essenziali del verismo:
    Uso di un punto di vista che permette al marratore di scomparire. Lo scrittore non deve mai esporre le proprie ideologie ed
    il suo punto di vista. Descrivono le pessime condizioni delle masse. L'uomo che nasce povero è destinato a rimanere
    povero. Non esiste la possibilità di fare una scalata sociale.

    Il destino per Verga porta solo sofferenza.
    Verismo collegato alla rivoluzione industriale.
    Ritenevano che un romando dovesse essere vero ed oggettivo.
    Zero sentimenti, valore scientifico.

    L'Italia era ancora un paese agricolo

    \section{Giovanni Verga}

    Era un pessimismo verista. La fede ridotta a semplice religosità. I monelli guardano a modelli perversi.

    Cavalleria Rusticana: i siciliani che chiamavano l'italia un continente e lui portava una pipa al re.
    Gli spacconi accendevano i fiammiferi con la scarpa.
    L'amore malato dalla gelosia porta all'odio.
    Il carrettiere si chiama Alfio.

\end{document}
