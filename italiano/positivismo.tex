\documentclass{article}

\usepackage{amsmath, mathtools, amsthm}
\usepackage{graphicx}
\graphicspath{ {./images/} }

\title{Leopardi}
\author{github.com/asdrubalini}
\date{\today}

\begin{document}
    \maketitle

    \section{Il Sabato del villaggio (1828)}
    La prima strofa è molto descrittiva.
    L'Idillio nella poesia greca è un componimento potetico di carattere pastorale.
    Leopardi prende spunto dalla poesia greca ma la amplia aggiungendo una sua riflessione sulla condizione esistenziale dell'uomo.
    \newline
    L'uomo confonde la speranza con i suoi desideri. Ognuno attraverso il suo intelletto farà ritorno al faticoso lavoro abituale.
    Man mano che passano le ore della domenica cresce l'ansia perchè ognuno torna al suo lavoro abituale. Poi si rivolge in maniera diretta
    a te, un giovane del 2021 che sta leggendo il testo e che non conosce ancora il male della vita. Tutti siamo in attesa di qualcosa, però
    quando quel qualcosa arriva effettivamente, ne restiamo delusi.
    La festa di cui parla Leopardi è l'età adulta, uscire di casa, andare via dall'oppressione dei propri genitori che impongono delle regole,
    avere un'indipendenza economica.
    \newline*
    Sembra un qualcosa di pesante, l'oppressione della scuola, l'ITIS che è un carcere, l'oppressione sistematica da parte dei genitori, senza
    rendersi conto che si sta vivendo in realtà il momento più bello, il sabato della vita. $\langle \langle$Ma sappi una questione, la tua festa (età adulta)
    anche se dovesse tardare a venire, non ti sia pesante.$\rangle \rangle$. Arriverà la festa e vi accorgerete che sarà tutt'altro.
    Da una descrizione ad una riflessione sulla vita di carattere generale.
    \newline
    Meditare fa rima con medicare, l'etimologia è la stessa. Le conclusioni di Leopardi appaiono corrette, pur nella loro malinconica concezione
    della vita dettata dal suo pragmaticismo materialista.
    \newline
    Esiste la felicità o vi è attesa di qualcosa che possa illuderci dell'attesa della felicità?
    Dolce illusione dei caratteri giovanili. La festa vera non è nella domenica ma nel sabato che la precede. Viene fuori il materialismo.
    Una grande similitudine con giovinezza (età della presunzione) ed età adulta. Uno pensa di comprendere tutto della vita.
    $\langle \langle$Fate si che la vostra giovinezza trascorra lentamente.$\rangle \rangle$
    \newline
    Condizione di pessimismo individuale.
    Idillio: componimento poetico di carattere pastorale / contadino + riflessione sulla condizione esistenziale dell'uomo.
    In Greco significa piccolo quadro / piccola immagine. Anche per Leopardi diventano un quadro sull'immagine della natura, quasi la
    contemplasse.
    Per l'infinito, molti critici hanno parlato addirittura di Idillio sacro. In fondo vi è comunque un'intuizione religiosa.
    Leopardi va oltre la natura. Dalla contemplazione si passa alla medicazione.
    \newline
    Che cos'è la vita per Leopardi? Lo troviamo nello Zibaldone, il suo diario personale. Lo Zibaldone non è un'opera vera e propria,
    ed è uscita dopo la sua morte. 15 anni di riflessioni, meditazioni, riflessioni su se stesso e gli altri, riflessioni sulla vita.
    Vien fuori la sua vita. Cantiere dove butta giù le idee. Non ha una logica, non si può leggere come se fosse.
    \newline
    Leopardi: $\langle \langle$Che cos'è la vita?$\rangle \rangle$. E poi parla di tutt'altro. Lo spiega attraverso una metafora. $\langle \langle$Il viaggio di uno zoppo infermo che con
    un gravissimo carico in sul dorso per montagne altissime e luoghi sommamente aspri, faticosi e difficili, alla neve, al gelo,
    alla pioggia, al vento, all'ardore del Sole, cammina senza mai riposarsi, dì e notte, uno spazio di molte giornate per arrivar a un
    cotal precipizio o fosso e quivi inevitabilmente cadere$\rangle \rangle$ (1826).
    Quello che i critici chiamano pessimismo cosmico.
    \newline
    Viaggio di una persona che soffre ed è infermo. La malattia dell'uomo è il materialismo, che però non viene detto esplicitamente.
    Il gravissimo carico che ogni uomo ha sul dorso è la ragione, che ci nobilita e ci può condannare se non correttamente intesa, come un dispetto.
    La vita è un viaggio in salita. Uno si accorge che sta vivendo non quando va in discesa ma quando inizia la salita ed inizia a scoprire tutte
    i problemi che gli si prospettano davanti. Non esiste un uomo che può dirsi esente dal far fatica. Neanche i più potenti al mondo.
    Tutti facciamo esperienza della fatica e di momenti difficili. Sia al gelo, sia quando tutto è ghiacciato.
    Camminiamo senza mai riposarci per arrivare di fronte ad un precipizio e cadere inevitabilmente dentro.
    Parte da una concezione razionale.
    \newline
    Se la vita ha senso, Leopardi ha torto. Ma se la vita non ha senso, ha ragione. La vita è solo materia oppure c'è altro?
    Un dono che non può essere eterno. Perchè abbiamo dentro di noi la coscenza dell'infinito e della realtà? (min 59) Pensare a qualcosa
    di eterno. Se l'uomo è finito, come fa a concepire l'infinito?
    Pensare a qualcosa che potesse durare. La vita è un tragico annasparsi verso la morte.

    \section{Recanati}
    Leopardi è nato a porto Recanati, un piccolo borgo di provincia in uno Stato retrogrado (dal punto di vista industriale). La famiglia di Leopardi era ricca. Non era bravo a gestire i soldi. Spendeva tutto in libri ed il bilancio della famiglia era andato in rosso.

    Progressista = opposto di conservatore
    Il padre di Leopardi era un conservatore anti-rivoluzionario.

    Tradizione con la T maiuscola. Esiste una catena ininterrotta nel tempo che parte da quando è stata inventata la scuola fino ad arrivare nei giorni nostri. Essere conservatori = tramandare ciò che si ha ricevuto.
    Con il progressismo invece si reputa tutto ciò che è passato vecchio per affermare un'idea di vecchio.

    \section{Casa di Leopardi}
    Palazzo Leopardi non era un palazzo da quattro soldi. Era una struttura immensa.
    Il patrimonio di casa Leopardi era completamente disastrato.

    Aveva sempre la gonna e gli stivaloni sotto. A quei tempi non andava di moda.
    lei era quella che portava i pantaloni in casa (per dire). Riuscì a sistemare le finanze in pochissimo tempo.
    Leopardi aveva un morbo che lo rendeva brutto. Le vertebre potrebbero collassare se non viene trattato. Il morbo non è altro che una tubercolosi extrapolmonare. Oltre ad avere la memoria eidetica aveva anche seri problemi fisici.

    Compito: breve ricerca tra conservatori e progressisti, non solo dal punto di vista politico.
    
    La mamma era così arretrata che Leopardi aveva persino paura di schiacciare la croce che si formava dall'incrocio delle mattonelle sul pavimento.

    Per sette lunghissimi anni si immerge in uno "studio matto e disperatissimo", sue parole riportate nello Zibaldone. Lo stare in casa accentuò il suo aspetto fisico, il morbo si sviluppò sempre di più. Acquistò una coonoscienza eccezionale delle lingue antiche e scrisse molte opere.

    Si innamorò di una donna di nome Teresa. Ad un certo punto muore e prende spunto da questa esperienza traumatica per parlarne all'interno dello Zibaldone. Secondo lui, aveva un non so che di divino che niente può eguagliare.

    Aveva quell'aria di innocenza, di ignoranza completa del male.
    Lo sguardo verso il basso, simbolo dell'innocenza.

    L'illuminismo è la luce che entra nella stranza buia della storia. Il romanticismo invece in un certo senso è un po' il capovolgimento perchè essere romantico vuol dire avere a che fare con una sfera non più razionale.

    Romanticismo: sprofondare nella paura sentimentale.

    Nel 1819 tenta di fuggire ma viene scoperto dai propri genitori. Tra il 1818 e il 1822 c'è la prima stagione poetica perchè scrive i piccoli Idilli.

    All'interno di un caffè di innamora di una donna che però amava solo i suoi scritti e non la persona. Era innamorata invece del suo amico.

    \section{}
    
    L'aggettivo dimostrativo questo si usa per indicare una cosa vicina a chi parla e lontana da chi ascolta. In questo caso colli e siepe sono vicini a Leopardi.

    Quello quando sono lontani.
    Dal punto di vista fisico, parliamo dell'ultimo orizzonte come il punto in cui la terra si congiunge con il cielo.

    Quante volte ci fermiamo all'ostacolo e non riusciamo ad andar oltre? Per noi quell'ostacolo diventa morte.
    
    Noi siamo fatti per qualcosa che va oltre questa sofferenza.

    "Mi sovvien l'eterno".

    Siamo in un'altra dimensione chiamata mistica contemplativa. Ecstasi.

    Diceva Giovanni Da Yebes "La via dell'unione con Dio che tanto desidero per tanto in questa via dell'unione, incominciare a camminare equivale ad abbandonare il proprio cammino o, per dire meglio, è un passare oltre fino a raggiungere la meta e liberarsi del proprio modo di agire vuol dire entrare in ciò che non ha modo, cioò il Dio".

    \section{}

    Recanati si trova vicino a Loreto.

    La via lauretana mette in collegamento Assisi con Loreto. L'ultimo orizzonte dal punto di vista geografico significa mirare ed osservare in profondità.

    "Sempre caro mi fu quest'ermo colle." Rompe lo schema temporale.

    "Questo" perchè Leopardi si riferisce ad una cosa a lui vicina.

    Primi tre versi dell'Infinito
    Sono tre versi liberi, la sua scelta è quella di rompere anche la metrica, non ha uno schema fisso, non ha uno schema rigido. 
    Non vi è un sistema metrico chiuso.

    "Fingo" vuol dire raffigurarsi in qualcosa perchè lui si immagina oltre la siepe.

    Caro = sia persona vicina a noi, sia carne
    Immagina quello che ci può essere oltre. Ma non può vederlo.

    Ma sedendo e mirando = Leopardi si siede e si immerge
    "Ma" nega quello che dice prima. Come a voler andare oltre quello stato.

    Ha bisogno di silenzio. Tende a rannicchiarsi un po' su sè stesso, un po' annichilito. Sedersi per guardare. "E io nel pensier mi fingo" per poter immaginare bisogna chiudere gli occhi. Si raccoglie in sè stesso, rientra in sè stesso, non si chiude. Cosa che noi non siamo più capaci di fare.

    Allegoricamente la siepe cosa rappresenta? La professoressa del Sarpi ha la definizione che c'è sui libri di testo.

    Con l'immginazione cercare di andare oltre, assopire gli occhi, chiudere ed attraverso l'immaginazione andare oltre nel silenzio. Profondissima quiete. tutto rientra un po' in sè stesso, come la tartaruga quando c'è un pericolo intorno. Però ad un certo punto proprio perchè inizia a fare un'esperienza che va oltre la semplice immaginazione ma arriviamo alla contemplazione, il cuore per poco non si spaura, prova uno smarrimento, un qualcosa che non aveva mai provato prima e che non riesce a definire. Sono manifestazioni che avvengono raramente e che colpiscono l'interno dell'uomo in maniera quais mistica, contemplativa. Dimensione in cui è difficile anche parlare.
    Difficile da descrivere, tanto che non riesce nemmeno Leopardi.

    La memoria è il ricettacolo di tutto quello che c'è. La memoria viene completamente assorbita da tutto quello che ci è capitato durante la giornata.

    Siamo sempre insoddisfatti, infelici, irrequieti.

    Ultimo orizzonte: punto dove la Terra si congiunge con il cielo.

    Fingo = latinismo, mi immagino, mi raffiguro.

    Interminati, immensi, senza termini, senza confini
    Sovrumani = sconosciuti all'uomo

    Va oltre la sfera sensoriale ed intellettiva.
    Questa siepe = vicina a chi parla e lontana da chi ascolta.

    Quel silenzio infinito di quegli spazi immensi di cui avevo fatto esperienza precedentemente vado paragonando a QUESTA voce. La voce delle piante.

    Attraverso l'immaginazione fa esperienza dell'infinito. Quando ritorna in sè stesso.
    "la voce del vento tra le piante".
    Come uno che classifichiamo pessimista cosmico possa dire "mi sovvien l'eterno".

    Ultimamente si è scoperto il chiaro richiamo ad un passo della Bibbia.

    Primo libro dei Re Capitolo 19 versetti 9-14

    La siepe è immagine di un'ostacolo, secondo la maggior parte dei critici è la morte.
    Sfera contemplativa, "si annega il pensier mio" e di conseguenza anche tutta la sfera sensoriale. Si va oltre la sfera sensoriale, il razionalismo che naufraga mi è dolce, è una cosa soave.

    Come si può dire "naufragare mi è dolce?". Quando uno naufraga fa l'esperienza del terrore, e invece lui no. Proprio perchè capisce che l'intelletto ormai è naufragata. La ragione diventa un dispetto ed un limite se non usata nel modo giusto. Si annega in questa intensità. Siamo a livello anagogico. Leopardi è andato oltre il suo modo di pensare ed agire.

    Collegamento con Giovanni da Yebes.

    Leopardi probabilmente ha fatto questa esperienza. Voleva stracciarla perchè non corrisponde a quelli che sono i gusti ed i canoni della società.

    \subsection{Operette morali}
    Dialoghi avvenuti tra personaggi storici ed inventati attraverso un racconto in prosa, con la struttura del dialogo. Qui vedremo Leopardi pessimista dal punto di vista cosmico. I dialoghi sono 24, 19 furono scritti nel 1824.

    Le operette morali non sono altro che una riflessione interiore. Il poeta si chiede il perchè della vita e dell'universo ma queste domande sono senza risposta.

    Moralità consiste nel guardare in faccia alla propria disperazione. Guardare in faccia gli spazi sterminati dell'universo e proclamare la propria dignità di uomo.

\end{document}
