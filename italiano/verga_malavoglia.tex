\documentclass{article}

\usepackage{amsmath, mathtools, amsthm}
\usepackage{graphicx}
\graphicspath{ {./images/} }

\title{Verga ed i Malavoglia}
\author{github.com/asdrubalini}
\date{\today}

\begin{document}
    \maketitle

    Verga sostiene che il progresso sia una fonte di sofferenza. I vinti sono quelli che non riescono a stare a passo con il progresso.
    Ideale dell'ostrica: se l'ostrica si stacca dallo scoglio muore. Se noi ci stacchiamo dalla famiglia e dalle tradizioni moriamo. Per non perderci dobbiamo restare ancorati al nostro sistema di valori.

    Verga per descrivere le classi povere della Sicilia ha bisogno di una serie di documenti. Nasce in Sicilia da una famiglia di proprietari terrieri, non ha mai provato sulla sua pelle quello di cui parla.

    I malavoglia, dei pescatori siciliani. Protagonisti sono i vinti. Il nome dell'intero ciclo sarebbe stato ciclo dei vinti, avrebbe dovuto rappresentare gli sconfitti nella lotta dell'affermazione sociale.

    Secondo Verga, al top della scala sociale ci sono gli artisti. I pescatori invece sono al gradino più basso

    La vicenza si svolge subito dopo l'unità d'Italia 1865-1875 ad Acitrezza in provincia di Catania. 

    Bastianaccio aveva sposato Maruzza la Longa

    Il giovane è costretto a partire per la leva militare. Padron ntoni decide di comprare a debito dei lupini per poi rivenderli. La barba però naufraga e muoiono in mare. Vende la barca e vende la casa per ripagare il debito dei lupini.


    Non c'è giudizio da parte dell'autore. Verga con questo racconto ci sta dicendo che compie un'indagine scientifica osservando la reazione al progresso da parte di gente umile. Come si sviluppano i primi turbamenti al pensiero di poter migliorare la propria condizione.

    Il progresso viene rappresentato con un fiume che scorre continuamente. Selezione naturale, c'è chi resiste e chi viene completamente soppiantato. (riga 5).
    Verga non andrà avanti con il giro dei vinti perchè diventa sempre più complesso.
    Nei malavoglia la ricerca del progresso è una lotta dei bisogni materiali.

    I personaggi si presentano da soli, non è il narratore a presentarli.
    Appena la famiglia Malavoglia prova ad arricchirsi hanno un debito e perdono tutto.

    \subsection{Verga e la Sicilia}
    Pag. 132
    Verga parla con nostalgia della sua Sicilia. Da giovane si trasferì a Milano.
    I malavoglia erano dei grandi lavoratori.
    La barca si chiamava provvidenza. In realtà la barba naufraga e bastianazzo muore. Da lì inizia il declino.


\end{document}
