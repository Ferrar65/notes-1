\documentclass{article}

\usepackage{amsmath, mathtools, amsthm}

\usepackage{graphicx}
\graphicspath{ {./images/} }

\title{Storia}
\author{github.com/asdrubalini}
\date{\today}

\begin{document}
    \maketitle

    \section{Rivoluzioni industriali}
    Ricerca sulla differenza tra la prima, seconda, terza e quarta rivoluzione industriale.

    Società di massa: la nostra società. Nella società attuale si è realizzata una diffusione di massa
    dei prodotti di consumo. Una società standardizzata ed omologata. Prodotti disponibili per un numero
    illimitato di persone. Sono alla portata di tutti nel mondo occidentale industrializzato.
    \newline
    I singoli individui scompaiono rispetto alla società, il gruppo. Ciò che si deve portare avanti è
    il nome della società, una società che non ha volto, che non è caratterizzata da persone. E nel nome 
    della società sono avvenute le più grandi aberrazioni.

    \section{Prima rivoluzione industriale}
    La Prima rivoluzione avvenuta dal 1780 al 1830 riguardò principalmente il settore produttivo tessile e metallurgico, la produzione divenne più veloce e semplice grazie alle nuove scoperte scientifiche finalizzate alla messa appunto di nuovi macchinari più efficienti azionati dalla macchina a vapore.Jul 29, 2021.

    \section{Seconda rivoluzione industriale}
    La 2a rivoluzione industriale (1870-1945)vide la fine del primato dell'Inghilterra e l'ascesa della Germania e degli Stati Uniti,si basò su due nuove forme di energia,l'elettricità e il petrolio,che sostituirono il carbone, e creò oggetti che trasformarono la vita quotidiana.

    \section{Terza rivoluzione industriale}
    La terza rivoluzione industriale portò alla scoperta e all'utilizzo di nuove fonti energetiche come l'atomo, gia usato nelle bombe atomiche del 1945 sul Giappone, all'energia solare, del vento, prendendo spunto dai mulini, all'aumento del consumo del petrolio e all'invenzione della plastica da parte di Moplen.

    \section{Quarta rivoluzione industriale}
    Per quarta rivoluzione industriale si intende la crescente compenetrazione tra mondo fisico, digitale e biologico. È una somma dei progressi in intelligenza artificiale (IA), robotica, Internet delle Cose (IoT), stampa 3D, ingegneria genetica, computer quantistici e altre tecnologie.Aug 14, 2019

    \section{Società di massa, avvento della seconda rivoluzione industriale}
    Da pag. 24 a pag. 32

    La società come la intendiamo oggi nasce con la seconda rivoluzione industriale. La società del pieno. Tutto è pieno. Trovare posto è diventato difficile in qualsiasi cosa.

    La massa è in'insieme omogeneo in cui i singoli individui scompaiono rispetto al gruppo. Vi è una vera e propria rivoluzione nel campo della sociologia. Dalle piccole comunità dove la persona veniva salvaguardata, ampliandosi il contesto la persona si smarrisce.
    Ormai l'economia di mercato è superata e va ripensata.

    Ricerca: Finestra di Overton. Un meccanismo psicologico al quale una persona cede lentamente alcune piccole libertà fino a concedere tutto.

    Con la terza rivoluzione industriale, la società di massa di distribuisce in tutto il pianeta dando origine alla globalizzazione.

    Nobiltà, clero e borghesia. Poi si aggiunge un nuovo stato, il proletariato.

\end{document}
