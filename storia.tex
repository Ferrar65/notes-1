\documentclass{article}

\usepackage{amsmath, mathtools, amsthm}

\usepackage{graphicx}
\graphicspath{ {./images/} }

\title{Storia}
\author{github.com/asdrubalini}
\date{\today}

\begin{document}
    \maketitle

    \section{Rivoluzioni industriali}
    Ricerca sulla differenza tra la prima, seconda, terza e quarta rivoluzione industriale.

    Società di massa: la nostra società. Nella società attuale si è realizzata una diffusione di massa
    dei prodotti di consumo. Una società standardizzata ed omologata. Prodotti disponibili per un numero
    illimitato di persone. Sono alla portata di tutti nel mondo occidentale industrializzato.
    \newline
    I singoli individui scompaiono rispetto alla società, il gruppo. Ciò che si deve portare avanti è
    il nome della società, una società che non ha volto, che non è caratterizzata da persone. E nel nome 
    della società sono avvenute le più grandi aberrazioni.

\end{document}
