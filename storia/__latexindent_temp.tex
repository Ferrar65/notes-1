\documentclass{article}

\usepackage{amsmath, mathtools, amsthm}

\usepackage{graphicx}

\title{Storia}
\author{github.com/asdrubalini}
\date{\today}

\begin{document}
    \maketitle

    \section{Rivoluzioni industriali}
    Ricerca sulla differenza tra la prima, seconda, terza e quarta rivoluzione industriale.

    Società di massa: la nostra società. Nella società attuale si è realizzata una diffusione di massa
    dei prodotti di consumo. Una società standardizzata ed omologata. Prodotti disponibili per un numero
    illimitato di persone. Sono alla portata di tutti nel mondo occidentale industrializzato.
    \newline
    I singoli individui scompaiono rispetto alla società, il gruppo. Ciò che si deve portare avanti è
    il nome della società, una società che non ha volto, che non è caratterizzata da persone. E nel nome 
    della società sono avvenute le più grandi aberrazioni.

    \section{Prima rivoluzione industriale}
    La Prima rivoluzione avvenuta dal 1780 al 1830 riguardò principalmente il settore produttivo tessile e metallurgico, la produzione divenne più veloce e semplice grazie alle nuove scoperte scientifiche finalizzate alla messa appunto di nuovi macchinari più efficienti azionati dalla macchina a vapore.Jul 29, 2021.

    \section{Seconda rivoluzione industriale}
    La 2a rivoluzione industriale (1870-1945)vide la fine del primato dell'Inghilterra e l'ascesa della Germania e degli Stati Uniti,si basò su due nuove forme di energia,l'elettricità e il petrolio,che sostituirono il carbone, e creò oggetti che trasformarono la vita quotidiana.

    \section{Terza rivoluzione industriale}
    La terza rivoluzione industriale portò alla scoperta e all'utilizzo di nuove fonti energetiche come l'atomo, gia usato nelle bombe atomiche del 1945 sul Giappone, all'energia solare, del vento, prendendo spunto dai mulini, all'aumento del consumo del petrolio e all'invenzione della plastica da parte di Moplen.

    \section{Quarta rivoluzione industriale}
    Per quarta rivoluzione industriale si intende la crescente compenetrazione tra mondo fisico, digitale e biologico. È una somma dei progressi in intelligenza artificiale (IA), robotica, Internet delle Cose (IoT), stampa 3D, ingegneria genetica, computer quantistici e altre tecnologie.Aug 14, 2019

    \section{Società di massa, avvento della seconda rivoluzione industriale}
    Da pag. 24 a pag. 32

    La società come la intendiamo oggi nasce con la seconda rivoluzione industriale. La società del pieno. Tutto è pieno. Trovare posto è diventato difficile in qualsiasi cosa.

    La massa è in'insieme omogeneo in cui i singoli individui scompaiono rispetto al gruppo. Vi è una vera e propria rivoluzione nel campo della sociologia. Dalle piccole comunità dove la persona veniva salvaguardata, ampliandosi il contesto la persona si smarrisce.
    Ormai l'economia di mercato è superata e va ripensata.

    Ricerca: Finestra di Overton. Un meccanismo psicologico al quale una persona cede lentamente alcune piccole libertà fino a concedere tutto.

    Con la terza rivoluzione industriale, la società di massa di distribuisce in tutto il pianeta dando origine alla globalizzazione.

    Nobiltà, clero e borghesia. Poi si aggiunge un nuovo stato, il proletariato.

    \section{Come fare una schemappa}

    La società di massa:
    Cos'è? Secondo il filosofo spagnolo Ortega y Gasset, società del pieno.
    Società del pieno -> Case, alberghi, ristoranti, treni, tutti pieni. Quello che prima non soleva essere un problema, ora lo è.

    La società di massa è la nostra società perchè.

    Siamo nel 1800. Si affacciano nuove fasce di popolazione che si iniziano ad occupare di politica. Proletariato = coloro che hanno solo la prole. Non hanno neanche gli occhi per piangere. 

    A cavallo tra il 1800 ed il 1900 è il secolo della nascita dei grandi partiti di massa. Ad esempio il grande modo il partito socialista in italia con conseguenti sindacati, volti a tutelare il lavoratore.

    Borghesia = appartenenti al ceto medio e che esecitano il commercio o una professione in proprio. I liberali volevano che lo Stato rimanesse fuori dall'economia, e che ne entrasse il meno possibile.

    Fatto che scoinvolge l'assetto della società: era possibile scalare le vette della società grazie al capitale posseduto. Da statica a dinamica. Inizialmente, chi nasceva povero non poteva arricchrsi, mentre nell'800 inizia a non essere più vero. Una rivoluzione.
    Con la seconda rivoluzione industriale c'è un nuovo protagonista: le masse.

    Nuovo personaggio: Teofilo Patini
    Dopo l'unificazione dell'Italia, quando Vittorio Emanuele II fu proclamato Re d'Italia, 

    I conservatori dal punto di vista economico volevano che la società rimanesse statica.

    Da ricordare che a Pio IV: Atto magisteriale dal quale i cattolici devono far riferimento, inviando a prendere una posizione chiara su determinate posizioni. Per i cattolici diventa vincolante.

    27 Settembre 1870 presa di Roma.

    E' un ingiustizia grande non dare all'operaio il dovuto guadagno.

    \section{}
    Nel corso dell'1800 all'interno del partito socialista si impose la tendenza marxista.

    Punti chiave:

    \begin{enumerate}
        \item Abbattere la borghesia
        \item Niente più distinzioni nelle classi sociali
        \item La proprietà privata è un furto e quindi va abolita
        \item Lo stato doveva garantire equilibrio ed uguaglianza tra sudditi
        \item Abbattimento in maniera violenta di ogni classe sociale
    \end{enumerate}

    Nel 1848 inizia il socialismo con Marx.
    Per un periodo di transizione si augurava la "dittatura del proletariato".
    In Italia il partito socialista viene fondato nel 1892 a Genova. Il suo principale esponente fu Filippo Turati che era più moderato rispetto ad altri che erano massimalisti.

    I riformisti sono quelli che vogliono entrare nel governo (Filippo Turati).
    La corrente massimalista era la corrente del comunismo da guerra (Benito Mussolini).

    Nel 1906 nacque il partito laburista in Gran Bretagna.

    "Socialisti di tutto il mondo, unitevi"

    \section{La dottrina sociale della chiesa cattolica: Mercoledì 27 Ottobre 2021}
    Il Rerum Novarum:

    \begin{enumerate}
        \item Denuncia gli eccessi del capitalismo
        \item Condanna le teorie socialiste e collettiviste
        \item Invita lo Stato ad intervenire
        \item Condanna la lotta delle classi
    \end{enumerate}

    \section{Belle Epoque}
    Nasce il secondo colonialismo o epoca degli imperi, meglio detta imperialismo.
    Il primo problema del colonialismo fu quello della tratta dei negri. Le popolazioni africane venivano prese da schiavisti arabi e venivano trapiantati nelle nuove americhe per lavorare.

    Fare una tabella sul primo, secondo e terzo colonialismo, con tutte le caratteristiche.
    Simbolo della Belle Epoque = Torre Eiffel, creata per l'Expo.

    Si riempivano di persone in cerca di distrazione e novità. Per quell'epoca vedere una donna che mostra il polpaccio era uno scandalo.

    Contenzioso: Relativo a una contesa giudiziaria.
    Pag. 55

    Titolo schemappa: Le illusioni della bella epoca
    Bella epoca da cosa è caratterizzata? Avvio di un'epoca di pace ed interesse, tante scoperte ed invenzioni, malattie curabili, la vita quotidiana viene modificata da numerose invenzioni tipo telefono.
    Poi aggiungere la considerazione che la povertà si allontana, ma poi subito dopo far sorgere una domanda: "ma fu realmente così?". Il nostro libro dice che è una definizione CURIOSA se si pensa che si trattò di un'epoca dove si diffusero nazionalismo, imperialismo e razzismo. Dall'imperialismo una freccia dove si dice "seconda colonizzazione/colonialismo" e dire le caratteristiche del secondo colonialismo. Da lì una freccia che mostra le differenze tra il primo ed il terzo colonialismo.
    Poi da lì ritornare a Belle Epoche, dire qualcosa di cos'era il mouierre rouge (luoghi di piacere, divertimento, si rimepivano di persone in cerca di distrazioni e libertà). Poi far partire un'altra dove si porta al militarismo dove si diceva anche "fenomeni in corso = moltiplicarsi dei contenziosi tra una nazione ed un'altra". Anche la corsa al riarmo "militarismo".
    Città simbolo: Parigi, simbolo: Torre Eiffel.
    Un'altra freccia sulla quale andremo a parlare dei diversi nazionalismi.

    \section{Dibattito politico nella società di massa}
    (schemappa su Whatsapp Bonse)

    \section{Dottrina sociale della Chiesa}

    Rerum Novarum - Papa Leone XI 20 Settembre 1870 la presa di Roma con la scomunica di casa Savoia.
    
    Sotto Rerum Novarum:
    \begin{enumerate}
        \item Denuncia degli eccessi del capitalismo
        \item Condanna delle teoria socialiste
        \item Si sostiene la proprietà privata in quanto diritto naturale
        \item Condanna della lotta di classe, si invita a collaborare padroni ed operai
        \item Legittimità delle organizzazioni sindacali e tra gli operai
        \item Si chiede allo Stato di intervenire per rimuovere le cause che possono esasperare il conflitto tra operai e padroni.
    \end{enumerate}

    La legge della domanda e dell'offerta soggiace ai limiti imposti dalla norma morale.

    Conseguenza di tutto questo $\rightarrow$ nel 1919 Don Luigi Sturzo fonda il partito popolare italiano.

    Primo colonialismo:
    Questa fase incomincia nel 1607 con la fondazione del primo insediamento permanente in America a Jamestown in Virginia che fu la prima colonia e finisce nel 1783 con il Trattato di Parigi,con la raggiunta di ben 13 colonie in tutto il nord America.

    Secondo colonialismo:
    Questa fase incomincia nel 1830 con l'inizio della conquista dell'Algeria e finisce nel 1859 con l'annessione di Saigon, interessò l'Algeria, il Vietnam, la Guiana orientale, il Senegal, il Gabon, le isole di Tahiti e la Riunione.

    Pag. 54-55 fatta
    
    \section{Nazionalismo}
    Pag. 56

    Principio di nazionalità: si afferma nel 1850, consapevolezza dell'identità culturale e storia del proprio popolo.
    Nazionalismo: derivazione negativa (si afferma tra la fine del 1800 e la metà del 1900)

    Dal punto di vista geografico non si può distinguere l'Europa dall'Asia.
    La nazione è destinata a seguire la legge dell'evoluzione, la legge del più forte. Darwinismo sociale.

    Papini: la guerra è l'unica igiene del mondo. I nazionalisti vogliono la guerra come strumento di miglioramento sociale. D'Annunzio dice che la forza è l'unica legge della natura.

    2 Settembre 1870, Otto Von Bismark occupa la Sarza e la Lorena, battaglia di Sedan, vittoria della Russia, Napoleone III è cosretto a fuggire e Guglielmo I nella stanza degli specchi viene fatto imperatore della Germania, nasce il II Reich.

    Pangermanesimo.

    Studiare e schemappa fino a pag. 59.

    Antisemitismo: contro i semi di Abramo, contro Israele.
    
    \section{Giustificazioni teoriche del razzismo}
    Verdi era contemporaneo con Wagner. Lo odiava a morte.\\
    Si provò a legare il razzismo alla scienza. Il suo successo fu dovuto ad un'isteria collettiva, si manifestò nella società di massa.

    Studiare appunti pag. 59-60
    La schemappa dovrà tenere conto di positivismo comte darwinismo sociale + Chamberline e giustificazione teorica dal punto di vista logico e scientifico del razzismo. Nitziesche e Wagner che hitler prenderà come modelli.

    \section{Giolitti}

    Non esisteva ancora il suffraggio universale. Solo censitario = in base al reddito. Solo borghesi medio-alti andavano a votare, solo liberali - massoni.

    Nel 1898 arriva una crisi del grano. Ci furono dimostrazioni popolari. Chi era al potere si interessava solo di chi aveva i soldi. Un governo normale avrebbe cercato di capire i motivi delle tensioni. Il generale beccaris proclamò lo stato di assedio e sparò con i cannoni sulla folla. Umberto I, che era per i diritti civili, proclamò beccaris con una medaglia all'onore. Si proclamò per la prima volta la pratica dell'ostruzionismo. Crollò il parlamento.\\
    29 luglio 1900 avvenne un episodio a Monza, il Re Umberto I fu assassinato da un anarchico.\\
    Zanardelli nel 1888 aveva eliminato la pena di morte in Italia. Fu chiamato al governo insieme a Giolitti, già implicato nello scandalo della banca romana.


    \section{Autoritarismo vs autorevolezza}

    Viviamo in un periodo in cui c'è autoritarismo.
    Giolitti aveva una visione liberale e lungimirante.

    Attuò un modo di fare politica chiamato "del doppio volto" = conservatore con l'aristocrazia e democratico con i proletari.

\end{document}
