\documentclass{article}

\usepackage{amsmath, mathtools, amsthm}
\usepackage{graphicx}
\graphicspath{ {./images/} }

\title{Prima guerra mondiale}
\author{github.com/asdrubalini}
\date{\today}

\begin{document}
    \maketitle

    \subsection{Cause politiche}
    \begin{enumerate}
        \item Germania voleva diventare prima potenza mondiale al posto dell'inghilterra.
        \item La Francia aveva il desiderio di vincere contro la Germania (rivincita).
        \item Crisi dell'impero ottomano, gli altri stati volevano espandersi. La prima fu l'Austria.
        \item Due blocchi militari contrapposti: triplice alleanza e triplice intesa (Germania, Austria, Italia vs Gran Bretagna, Francia e Russia)
    \end{enumerate}

    \subsection{Cause economiche}
    \begin{enumerate}
        \item Rivalità tra le colonie
    \end{enumerate}

    \subsection{Cause militari}
    \begin{enumerate}
        \item Corsa agli armamenti spinta dai gruppi industriali
    \end{enumerate}

    \subsection{Cause culturali}
    \begin{enumerate}
        \item Volontà di affermare la propria nazione sulle altre
        \item La guerra decide le nazioni dominatric
        \item Futurismo
    \end{enumerate}

    \subsection{Inizio guerra}
    28 Giugno 1914: arciduca Francesco Ferdinando, erede al trono Austriaco, viene ucciso.
    L'Austria chiede un ultimatum alla Serbia chiedendo di consegnare l'assassino. La Serbia respinge poichè aveva stretto
    un accordo con la Russia.

    28 Luglio 1914: Austria e Germania vs Serbia, Russia, Francia ed inghilterra.
    Si pensava ad una guerra lampo per via degli aerei e delle trincee, una guerra di logoramento, ma non fu così.

    L'Italia si divide in neutralisti ed interventisti.
    Opinione pubblica, liberali (Giolitti), cattolici e la maggior parte dei socialisti non volevano entrare in guerra.
    Nazionalisti, irredentisti, gli industriali volevano entrare ed una parte dei socialisti nazionalisti.

\end{document}
