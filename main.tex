\documentclass{article}

\usepackage{amsmath, mathtools, amsthm}

\usepackage{graphicx}
\graphicspath{ {./images/} }

\title{Fibra ottica}
\author{github.com/asdrubalini}
\date{\today}

\begin{document}
    \maketitle

    \section{Utilizzo}

    La fibra ottica è un mezzo vetroso che convoglia un raggio luminoso al suo interno. 
    Il raggio può essere trasmesso solo se il materiale di cui è composto la fibra ha delle caratteristiche
    tali da risultare in una riflessione totale.

    \section{Legge di Snell}

    Esiste una relazione tra l'angolo di incidenza, l'angolo con cui viene rifratto il raggio nel secondo mezzo e gli indici di
    rifrazione dei due mezzi. La relazione è descritta dalla legge di Snell:

    \begin{equation}
        \frac{sen (\phi i)}{sen (\phi R)} = \frac{n_2}{n_1}
    \end{equation}

    Dalla formula si capisce che, aumentando l'angolo di incidenza, aumenta anche quello di rifrazione. Quando l'angolo di rifrazione
    raggiunge i 90 gradi, il segnale viene riflesso completamente. L'angolo di incidenza necessario per questa evenienza si chiama
    angolo limite $\phi L$ e si trova con la formula inversa della legge di Snell:

    \begin{equation}
        sen (\phi L) = \frac{n_2}{n_1}
    \end{equation}

    \section{Apertura numerica}

    L'apertura numerica è un parametro che caratterizza l'accoppiamento della fibra con la sorgente di radiazione.

    \begin{equation}
        NA = n_1 sin(\phi M) = \sqrt{n_1^2 - n_2^2}
    \end{equation}

    \section{Angolo di accettazione}

    Il segnale deve entrare nella fibra con un certo angolo, definito angolo di accettazione. L'angolo di accettazione si
    può ricavare dall'apertura numerica:

    \begin{equation}
        \phi M = arcsin(NA)
    \end{equation}

    \section{Modi di propagazione}

    Dati i parametri della fibra, possono esistere diversi raggi luminosi che la attraversano con percorsi diversi, distanze diverse
    e tempi diversi. Il numero di questi raggi è definito come modi di propagazione e si calcola con la seguente equazione:

    \begin{equation}
        M = \frac{1}{2} (\frac{\pi \cdot d \cdot NA}{\lambda})^2
    \end{equation}


\end{document}
