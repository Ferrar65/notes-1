\documentclass{article}

\usepackage{amsmath, mathtools, amsthm}

\usepackage{graphicx}
\graphicspath{ {./images/} }

\title{Fibra ottica}
\author{github.com/asdrubalini}
\date{\today}

\begin{document}
    \maketitle

    \section{Teoria}

    \subsection{Utilizzo}

    La fibra ottica è un mezzo vetroso che convoglia un raggio luminoso al suo interno. 
    Il raggio può essere trasmesso solo se il materiale di cui è composto la fibra ha delle caratteristiche
    tali da risultare in una riflessione totale.

    \subsection{Legge di Snell}

    Esiste una relazione tra l'angolo di incidenza, l'angolo con cui viene rifratto il raggio nel secondo mezzo e gli indici di
    rifrazione dei due mezzi. La relazione è descritta dalla legge di Snell:

    \begin{equation}
        \frac{sen (\phi i)}{sen (\phi R)} = \frac{n_2}{n_1}
    \end{equation}

    Dalla formula si capisce che, aumentando l'angolo di incidenza, aumenta anche quello di rifrazione. Quando l'angolo di rifrazione
    raggiunge i 90 gradi, il segnale viene riflesso completamente. L'angolo di incidenza necessario per questa evenienza si chiama
    angolo limite $\phi L$ e si trova con la formula inversa della legge di Snell:

    \begin{equation}
        sen (\phi L) = \frac{n_2}{n_1}
    \end{equation}

    \subsection{Apertura numerica}

    L'apertura numerica è un parametro che caratterizza l'accoppiamento della fibra con la sorgente di radiazione.

    \begin{equation}
        NA = n_0 sin(\phi M) = \sqrt{n_1^2 - n_2^2}
    \end{equation}

    \begin{equation}
        NA = n_1 cos(\phi L)
    \end{equation}

    \subsection{Angolo di accettazione}

    Il segnale deve entrare nella fibra con un certo angolo, definito angolo di accettazione. L'angolo di accettazione si
    può ricavare dall'apertura numerica:

    \begin{equation}
        \phi M = arcsin(NA)
    \end{equation}

    \subsection{Modi di propagazione}

    Dati i parametri della fibra, possono esistere diversi raggi luminosi che la attraversano con percorsi diversi, distanze diverse
    e tempi diversi. Il numero di questi raggi è definito come modi di propagazione e si calcola con la seguente equazione:

    \begin{equation}
        M = \frac{1}{2} (\frac{\pi \cdot d \cdot NA}{\lambda})^2
    \end{equation}

    \subsection{Distanza massima di un raggio}

    Ogni modo di propagazione corrisponde ad un raggio che percorre un percorso diverso all'interno
    della fibra, ed essendo la velocità sempre la stessa, le distanze possono variare. La distanza minima
    corrisponde alla lunghezza della fibra, mentre la distanza massima si può calcolare con gli indici
    di rifrazione di nucleo e mantello.

    \begin{equation}
        d_{max} = \frac{l \cdot n_1}{n_2}
    \end{equation}

    \subsection{Lunghezza d'onda di taglio di una fibra monomodale}

    Una fibra monomodale ha un solo modo che può essere propagato. Esiste una relazione tra il diametro del nucleo, l'apertura
    numerica e la lunghezza d'onda limite o critica $\lambda c$.

    \begin{equation}
        d = 0.76 \frac{\lambda c}{NA} \hspace{0.5cm} [m]
    \end{equation}

    \subsection{Banda di una fibra}

    La banda di una fibra si può calcolare sapendo la banda modale $B_m$ e la banda cromatica $B_c$.

    \begin{equation}
        B = \frac{1}{
            \sqrt{
                \frac{1}{B_m}^2 +
                \frac{1}{B_c}^2
            }
        } \hspace{0.5cm} [MHz]
    \end{equation}

    Banda modale:

    La dispersione modale è nulla per le fibre monomodali    

    \begin{equation}
        \Delta tm_o = 3333 \cdot \frac{n_1}{n_2} \cdot (n_1 - n_2) \hspace{0.5cm} [\frac{ns}{km}]
    \end{equation}

    \begin{equation}
        Bm_0 = \frac{0.44 \cdot 10^3}{\Delta tm_0} \hspace{0.5cm} [MHz \cdot km]
    \end{equation}

    \begin{equation}
        Bm = \frac{Bm_0}{l^{0.85}} \hspace{0.5cm} [MHz]
    \end{equation}
    
    Banda cromatica:

    \begin{equation}
        \Delta tc_o = \mu \cdot \Delta \lambda \hspace{0.5cm} [\frac{ps}{km}]
    \end{equation}

    \begin{equation}
        Bc_0 = \frac{0.44 \cdot 10^6}{\Delta tc_0} \hspace{0.5cm} [MHz \cdot km]
    \end{equation}

    \begin{equation}
        Bc = \frac{Bc_0}{l} \hspace{0.5cm} [MHz]
    \end{equation}

    \subsection{Scarto relativo degli indici di rifrazione}

    \begin{equation}
        S = \frac{n_1^2 - n_2^2}{2 \cdot n_1^2} \cdot 100
    \end{equation}

    \subsection{Perdite intrinseche ed estrinseche}

    Le perdite intrinseche di una fibra sono quelle che dipendono dalla tecnologia, mentre quelle estrinseche dipendono da
    difetti di produzione.

    Intrinseche, per scattering:
    \begin{equation}
        P = \frac{0.85}{\lambda ^ 4}
    \end{equation}

    Estrinseche:

    Tronchi con diametro diverso:
    \begin{equation}
        A_d = 20 \cdot log(\frac{d_s}{d_r}) \hspace{0.5cm} [dB]
    \end{equation}

    Tronchi con apertura numerica diversa:
    \begin{equation}
        A_NA = 20 \cdot log(\frac{NA_s}{NA_r}) \hspace{0.5cm} [dB]
    \end{equation}

    Tronchi con indice di rifrazione diversi:
    \begin{equation}
        \tau = \frac{4}{
            2 + \frac{n_1s}{n_1r} + \frac{n_1r}{n_1s}
        }
    \end{equation}
    \begin{equation}
        A_N = 10 \cdot log(\frac{1}{\tau}) \hspace{0.5cm} [dB]
    \end{equation}

\end{document}
