\documentclass{article}

\usepackage{amsmath, mathtools, amsthm}

\usepackage{graphicx}

\title{Networking}
\author{github.com/asdrubalini}
\date{\today}

\begin{document}
    \maketitle

    \section{TCP e UDP}
    TCP: Orientato alla connessione
    UDP: Non richiede una connessione

    Protocollo TFTP su rete locale non poggia su TCP ma su UDP perchè la rete locale ha una bassa latenza e una bassa probabilità di errori.

    Anche il protocollo DNS su rete locale usa UDP, mentre su internet usa TCP.
    Due server DNS tra loro non dialogano un UDP ma in TCP, sempre sulla porta 53.

    Il DNS è un protocollo critico perchè i DNS di tutto il mondo devono parlarsi e scambiarsi informazioni. I regimi dittatoriali configurano i DNS presenti sul territorio in modo che non diano tutte le informazioni.

    Ognuno si deve fidare delle informazioni che gli vengono rilasciate dall'altro.
    Ci sono delle autorità superiori che controllano chi possiede i DNS. Se si verificano delle anomalie, le autorità riescono a risalire a chi ha creato le anomalie.

    Socket = IP + Porta
    L'header del TCP normalmente ha 20 bytes + da 0 a 32 bytes opzionali.

    Sequence number = TCP tiene sotto controllo il flusso e da un ordine ai pacchetti.
    Acknowledgement number = chi riceve i dati comunica a chi li ha trasmessi il corretto recapito dei dati.

    Bit di controllo = quali funzioni sono attive in quel pacchetto. Ad esempio, se il pacchetto è parte di un segmento fragmentato oppure no. Oppure se il pacchetto contiene oppure no un ACK number.

    Window = delimitare il numero massimo di pacchetti che possono essere spediti senza ricevere un ACK.

    Checksum = controllo degli errori nella trasmissione (sia header che dati)

    Selective ACK = vengono reinviati solo i segmenti persi e non tutti, come è di default.

    Durante la fase di connessione (3 way handshake) si settano anche alcuni parametri della comunicazione che possono essere risettati durante la connessione. Ad esempio windows size dove ci si mette d'accordo sul numero massimo di segmenti che possono essere spediti prima di ricevere l'ACK.

    Se in un determinato momento ci sono tanti host che parlano con un server, il suo buffer si riempie e il server manda dei segmenti chiedendo di ridurre la finestra dei pacchetti massimi che si possono spedire ogni ACK.

    I protocolli di oggi utilizzano le sliding windows e inviano un ACK ogni due segmenti che ricevono.

    Le dimensioni della finestra non vengono definite in termini di segmenti ma in termini di bytes, quindi la finestra viene modificata ogni due segmenti.

    Tipicamente la dimensione massima di un segmento è di 1460 bytes ma può essere modificato nel corso della connessione.

    Alla fine dipende sempre dai tempi.

    Guardare a casa il capitolo 15.

    \section{Dominio di coalisione}

    Dominio di coalisione è l'insieme delle linee di una LAN in cui può esserci coalisione fisica = due dispositivi che inviano dei dati insieme.
    Ecludendo le connessioni wireless, sulle LAN con connessioni fisiche, il dominio di coalisione è il singolo tratto che connette i PC allo switch. La coalisione viene evitata dal fatto che la tipologia di connessione è in full duplex (però essendo in full duplex non c'è comunque collisione).

    Il dominio di broadcast è l'insieme dello spazio di rete raggiungibile dai pacchetti broadcast.
    Le reti VLAN servono a segmentare le reti in modo da ridurre i domini di broadcast.
    Un altro modo per segmentare una rete è quello di utilizzare un router con più interfacce e su ogni interfaccia viene suddivisa un'intera rete in sottoreti.

    I router aziendali ormai non hanno più tante interfacce perchè con il meccanismo della VLAN. Normalmente un rouer aziendale non ha più di 4 interfacce. Un paio vengono usate per il collegamento ad internet. Un'interfaccia per le sottoreti interne ed eventualmente un'altra interfaccia di collegamento diretto verso le altre filiali.

    Le reti VLAN danno molta flessibilità.

    La virtualizzazione non ha bisogno degli switch layer 3, sono sufficienti quelli di layer 2. Per configurarlo si va sugli switch e si crea una nuova VLAN con un certo nome. Poi comincio a lavorare sulla porta o sul range di porte che voglio usare per una determinata VLAN. Ad esempio ho uno switch 24 porte, su 12 voglio avere una VLAN e sulle altre 12 un'altra. Dichiaro le due VLAN, prendo il range delle prime 12 porte e faccio il tag della prima. Poi prendo l'altro range e faccio il tag della seconda.

    Esercizio: tre VLANs, un PC per ogni VLAN, un solo switch. Sulla VLAN 1 mettiamo l'indirizzo IP dello switch che fa da SVI (Switch Virtual Interface) con tre PC. Quindi la VLAN 1 non verrà mai usata per creare una VLAN. Per questo motivo di parte della VLAN 2.

    Di default le porte sono impostate per non far passare le VLAN. Posso dire allo switch di far transitare tutte le VLAN su una porta.

    in una delle tre vlan bisogna mettere due pc. poi bisogna aggiungere un altro pc che non appartiene a nessuna vlan così facciamo tutte le prove. 

    \section{VLAN Giovedì 28 Ottobre 2021}

    Una porta può essere taggata solo con una VLAN o con tutte.
    Il trunk permette il passaggio di tutte le VLAN. Non permette il passaggio di comunicazioni non taggate.

    Esempio: un'azienda dove faccio 3 VLAN per 3 reparti differenti e poi in una quarta VLAN metto dei server. I server presumibilmente dovranno comunicare con le altre tre VLAN.

    Per far comunicare due VLAN serve un Router. Sul router ci sarà bisogno di una sola interfaccia fisica.
    Su ogni interfaccia virtualizzo una scheda di rete.
    Si possono creare sottointerfacce.

    interface gigabitEthernet 0/0/0.1

    Il protocollo da usare è speciale perchè deve trattare dei pacchetti che a livello 2 hanno un tag.
    Usiamo lo standard 802.1Q. Il comando da usare è encapsulation.

    Ricapitolando:
    Una VLAN deve fare capo ad un'interfaccia virtuale del router.
    Entro nell'interfaccia virtuale del router e gli dico che lì arriveranno dei pacchetti taggati della VLAN.

    encapsulation dot1q <vlan id>

    Per far comunicare la VLAN 2 con la VLAN 3 devo andare sulla stessa interfaccia ed aprire una seconda interfaccia virtuale.

    show ip interface brief

    Mettere in comunicazione la VLAN 2 e la VLAN 3

    \section{Dibattito politico nella società di massa}
    (schemappa su Whatsapp Bonse)

    \section{Dottrina sociale della Chiesa}

    Rerum Novarum - Papa Leone XI 20 Settembre 1870 la presa di Roma con la scomunica di casa Savoia.
    
    Sotto Rerum Novarum:
    \begin{enumerate}
        \item Denuncia degli eccessi del capitalismo
        \item Condanna delle teoria socialiste
        \item Si sostiene la proprietà privata in quanto diritto naturale
        \item Condanna della lotta di classe, si invita a collaborare padroni ed operai
        \item Legittimità delle organizzazioni sindacali e tra gli operai
        \item Si chiede allo Stato di intervenire per rimuovere le cause che possono esasperare il conflitto tra operai e padroni.
    \end{enumerate}

    La legge della domanda e dell'offerta soggiace ai limiti imposti dalla norma morale.

    Conseguenza di tutto questo $\rightarrow$ nel 1919 Don Luigi Sturzo fonda il partito popolare italiano.

    Primo colonialismo:
    Questa fase incomincia nel 1607 con la fondazione del primo insediamento permanente in America a Jamestown in Virginia che fu la prima colonia e finisce nel 1783 con il Trattato di Parigi,con la raggiunta di ben 13 colonie in tutto il nord America.

    Secondo colonialismo:
    Questa fase incomincia nel 1830 con l'inizio della conquista dell'Algeria e finisce nel 1859 con l'annessione di Saigon, interessò l'Algeria, il Vietnam, la Guiana orientale, il Senegal, il Gabon, le isole di Tahiti e la Riunione.

    Pag. 54-55 fatta
    
    \section{Nazionalismo}
    Pag. 56

    Principio di nazionalità: si afferma nel 1850, consapevolezza dell'identità culturale e storia del proprio popolo.
    Nazionalismo: derivazione negativa (si afferma tra la fine del 1800 e la metà del 1900)

    Dal punto di vista geografico non si può distinguere l'Europa dall'Asia.
    La nazione è destinata a seguire la legge dell'evoluzione, la legge del più forte. Darwinismo sociale.

    Papini: la guerra è l'unica igiene del mondo. I nazionalisti vogliono la guerra come strumento di miglioramento sociale. D'Annunzio dice che la forza è l'unica legge della natura.

    2 Settembre 1870, Otto Von Bismark occupa la Sarza e la Lorena, battaglia di Sedan, vittoria della Russia, Napoleone III è cosretto a fuggire e Guglielmo I nella stanza degli specchi viene fatto imperatore della Germania, nasce il II Reich.

    Pangermanesimo.

    Studiare e schemappa fino a pag. 59.

    Antisemitismo: contro i semi di Abramo, contro Israele.

\end{document}
